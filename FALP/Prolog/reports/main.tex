\documentclass[12pt,a4paper,oneside]{report}
\usepackage[utf8]{inputenc}
\usepackage[english,russian]{babel}
\usepackage{amsmath}
\usepackage{amssymb}
\usepackage{geometry}
\usepackage{titlesec, blindtext, color}
\usepackage{tabularx}
\usepackage{subcaption}
\usepackage{listings}
\usepackage{tikz,pgfplots}
\geometry{pdftex, left = 2cm, right = 2cm, top = 2.5cm, bottom = 2.5cm}
% Для листинга кода:
\lstset{
	basicstyle=\footnotesize\ttfamily,
	breakatwhitespace=true,
	breaklines=true,
	commentstyle=\color{gray},
	frame=single,
	keywordstyle=\color{blue},
	stringstyle=\color{red},
	tabsize=8
}
\titleformat{\chapter}[hang]{\Huge\bfseries}{\thechapter.\textcolor{gray75}\hsp}{0pt}{\Huge\bfseries}
\newcommand{\specchapter}[1]{\chapter*{#1}\addcontentsline{toc}{chapter}{#1}}
\newcommand{\specsection}[1]{\section*{#1}\addcontentsline{toc}{section}{#1}}
\newcommand{\specsection}[1]{\section*{#1}\addcontentsline{toc}{section}{#1}}
\geometry{pdftex, left = 2cm, right = 2cm, top = 2.5cm, bottom = 2.5cm}
\begin{document}
\thispagestyle{empty}
\noindent \begin{minipage}{0.15\textwidth}
	\includegraphics[width=\linewidth]{b_logo}
\end{minipage}
\noindent\begin{minipage}{0.9\textwidth}\centering
	\textbf{Министерство науки и высшего образования Российской Федерации}\\
	\textbf{Федеральное государственное бюджетное образовательное учреждение высшего образования}\\
	\textbf{«Московский государственный технический университет имени Н.Э.~Баумана}\\
	\textbf{(национальный исследовательский университет)»}\\
	\textbf{(МГТУ им. Н.Э.~Баумана)}
\end{minipage}
\noindent\rule{18cm}{3pt}
\newline\newline
\noindent ФАКУЛЬТЕТ $\underline{\textbf{«ИНФОРМАТИКА И СИСТЕМЫ УПРАВЛЕНИЯ»}}$ \newline\newline
\noindent КАФЕДРА $\underline{\textbf{«ПРОГРАММНОЕ ОБЕСПЕЧЕНИЕ ЭВМ И ИНФОРМАЦИОННЫЕ }$\newline\newline $\underline{\textbf{ТЕХНОЛОГИИ»(ИУ7)}}}$\newline\newline
\noindent НАПРАВЛЕНИЕ ПОДГОТОВКИ $\underline{\textbf{09.03.04 ПРОГРАММНАЯ ИНЖЕНЕРИЯ}}$\newline\newline\newline\newline\newline\newline\newline
\begin{center}
    \begin{flushright}
    \Large\textbf{ОТЧЕТ}\newline
	\Large\textbf{по лабораторной работе № 13}\newline
	\end{flushright}
\end{center}
\noindent\textbf{Дисциплина:} $\underline{\text{Функциональное и логическое программирование}}$\newline\newline\newline\newline\newline\newline\newline\newline
\begin{tabular}{lcp{5em}lp{2em}l}
	\noindent\textbf{Студент} &  $\underline{\text{ИУ7-62Б~~}}$ &             &\hspace{1cm} & & $\underline{\text{Б.М.Блохин}}$ \\\cline{4-3}
	 & (Группа) & &(Подпись,дата)  & & (И.О.Фамилия) \\
	 & & & & &\\
	\noindent\textbf{Преподаватель} &  & &\hspace{1cm} & &$\underline{\text{Н.Б.Толпинская~~~~}}$ \\\cline{4-3} 
	 &  & & (Подпись,дата)  & &(И.О.Фамилия) \\
    \end{tabular}
\begin{center}
	\vfill
	Москва,\the\year
\end{center}
\newpage


\chapter*{Задачи}
\quad Составить программу, т.е. модель предметной области – базу знаний, объединив в ней информацию – знания:

«Телефонный справочник»: Фамилия, №тел, Адрес – структура (Город, Улица, №дома, №кв),

«Автомобили»: Фамилия\_владельца, Марка, Цвет, Стоимость, и др.,

«Вкладчики банков»: Фамилия, Банк, счет, сумма, др.

Владелец может иметь несколько телефонов, автомобилей, вкладов (Факты).
Используя правила, обеспечить возможность поиска:

а) По № телефона найти: Фамилию, Марку автомобиля, Стоимость автомобиля (может быть несколько),

в) Используя сформированное в пункте а) правило, по № телефона найти: только Марку автомобиля (автомобилей может быть несколько),

Используя простой, не составной вопрос: по Фамилии (уникальна в городе, но в разных городах есть однофамильцы) и Городу проживания найти: Улицу проживания, Банки, в которых есть вклады и №телефона.

Для задания1 и задания2: 

Для одного из вариантов ответов, и для а) и для в), описать словесно порядок поиска ответа на вопрос, указав, как выбираются знания, и, при этом, для каждого этапа унификации, выписать подстановку – наибольший общий унификатор, и соответствующие примеры термов.


\begin{lstlisting}[language=Prolog]
domains
surname, telephone, city, street = symbol.
home_number, flat_number = integer.
address = address(city,
street,
home_number,
flat_number).

car_brand, car_color = symbol.
car_price, car_year = integer.
about_car = about_car(car_brand,
car_color,
car_price,
car_year).

bank_name = symbol.
bank_account, bank_amount = integer.
dep_info = dep_info(bank_name,
bank_account,
bank_amount).

predicates
phone_record(surname, telephone, address).
car_owner(surname, about_car).
bank_owner(surname, dep_info).
car_by_number(telephone, surname, car_brand, car_price).
car_brand_by_number(telephone, car_brand).
info_by_surname_city(surname, city, street, bank_name, telephone).

clauses
phone_record("Gerasimenko", "85886893800", address("Moscow", "Auf", 17, 33)). 
phone_record("Kryat", "85886893892", address("Moscow", "Kolokol", 12, 33)). 
phone_record("Popov", "89932110022", address("Minsk", "Moskovskaya", 94, 11)). 
phone_record("Blohin", "89265291378", address("Moscow", "Varsh", 52, 9)).
phone_record("Blohin", "89115291378", address("Saratov", "Moskovskaya", 5, 9)).
car_owner("Paklin",
about_car("Cooper", "Red", 2000001, 2012)).
car_owner("Kryat",
about_car("Mitsubishi", "Grey", 289731, 2006)).
car_owner("Blohin",
about_car("Opel", "Grey", 290000, 2007)).
car_owner("Blohin",
about_car("Audi", "Red", 2900000, 2015)).
bank_owner("Kolosov",
dep_info("Sber", 1, 100000)).
bank_owner("Kolosov",
dep_info("VTB", 5, 200000)).
bank_owner("Popov",
dep_info("Rocket", 2, 10000000)).
bank_owner("Blohin",
dep_info("Gazprom", 4, 400000)).

car_by_number(Phone_num, Surname, Car_brand, Car_price):-
phone_record(Surname, Phone_num,_),
car_owner(Surname, about_car(Car_brand, _, Car_price, _)).

car_brand_by_number(Phone_num, Car_brand):-
car_by_number(Phone_num, _, Car_brand, _).

info_by_surname_city(Surname, City, Street, Bank, Phone_num):-
phone_record(Surname, Phone_num, address(City, Street, _, _)),
bank_owner(Surname, dep_info(Bank, _, _)).

goal
%car_by_number("89265291378", Surname, Car_brand, Car_price).
%car_brand_by_number("85886893892", Car_brand).
info_by_surname_city("Blohin", "Saratov", Street, Bank, Phone_num).
\end{lstlisting}

\chapter*{Контрольные вопросы}
\subsection*{Что такое терм?}
\quad Термы – слова, которые описывают сущности изучаемого мира.
\subsection*{Что такое предикат в матлогике (математике)?}
\quad Предикат – функция с множеством значений ${0, 1}$, определенная на множестве $M = M_1 * M_2 * … * M_n$.
\subsection*{Что описывает предикат в Prolog?}
\quad Предикаты – слова, которые описывают атрибуты/свойства сущностей, их поведение и отношение.
\subsection*{Виды предложений в программе, примеры таких предложений.}
\quad Факты – утверждения, которые всегда истинны. Предложения с пустым телом.

Правила- утверждения, истинность которых зависит от некоторых условий. Имеют голову и непустое тело.

Вопросы – с их помощью пользователь спрашивает систему о том, какие утверждения являются истинными. Предложения, состоящие только из тела.
\subsection*{Какие предложения являются основными, а какие – не основными?}
\quad Если составные термы, факты, правила и вопросы не содержат переменных, то они называются основными. 

Составные термы, факты, правила и вопросы в момент фиксации в программе могут содержать переменные, тогда они называются неосновными.

\subsection*{Каков синтаксис этих предложений} 
\quad Переменные обозначаются идентификаторами, начинающимися с заглавной буквы. 
\subsection*{Каково назначение переменных}
\quad Переменные обозначают некоторый неизвестный объект из некоторого множества объектов. 
\subsection*{Особенности использования переменных в программе на Prolog?}
\quad В момент фиксации утверждений в программе не имеют значения. Значения для переменных могут быть установлены системой только в процессе поиска ответа на вопрос, т.е. реализации программы.
\subsection*{Виды переменных}
\begin{itemize}
    \item Именованные – есть имя – комбинация символов.
    \item Анонимные – нет имени – символ подчеркивания.
    \item Связанная (конкретизирована) – имеется объект, который в данный момент обозначает данная переменная.
    \item Свободная (не конкретизирована)
\end{itemize}

\subsection*{Какое предложение БЗ сформулировано в более общей – абстрактной форме: содержащее или не содержащее переменных?}
\quadНе содержащее переменных.
\subsection*{Что такое подстановка?}
\quadПодстановка - множество пар вида ${X_i = t_i}$, где $X_i$ – переменная, а $t_i$ – терм.
\subsection*{Что такое пример терма?}
\quad Терм В называется примером терма А, если существует такая подстановка Θ, что B = $A\Theta$, где $A\Theta$ – результат применения подстановки к терму. 
\subsection*{Как и когда строится?}
\quad Примеры строятся во время алгоритма унификации.



\end{document}